\documentclass[12pt]{article}

% --- Página y tipografía ---
\usepackage[letterpaper,margin=2.5cm]{geometry}
\usepackage[T1]{fontenc}
\usepackage[utf8]{inputenc} % si compilas con pdfLaTeX
\usepackage{lmodern}
\usepackage{microtype}

% --- Imágenes y color ---
\usepackage{graphicx}
\usepackage{xcolor}

% --- Control fino de espacios ---
\usepackage{setspace}
\setlength{\parindent}{0pt}

\begin{document}
\thispagestyle{empty}

% ===== Encabezado con logos + texto =====
\begin{minipage}[c]{0.18\textwidth}
    \centering
    % Cambia por tu logo izquierdo
    \includegraphics[width=0.95\linewidth]{img/logo_usac.jpeg}
\end{minipage}
\hfill
\begin{minipage}[c]{0.60\textwidth}
    \small
    Universidad de San Carlos de Guatemala\\
    Escuela de Ciencias Físicas y Matemáticas\\
    Nombre estudiante: Ilsy Estefania Castillo Ortiz\\
    Carnet: 202005371\\
    Programación 1\\
\end{minipage}
\hfill
\begin{minipage}[c]{0.18\textwidth}
    \centering
    % Cambia por tu logo derecho
    \includegraphics[width=1.4\linewidth]{img/logo_ecfm.jpg}
\end{minipage}

\vspace{0.5cm}

% Línea horizontal superior (gruesa)
\noindent\rule{\textwidth}{1.2pt}

\vspace{0.2cm}

% ===== Título =====
\begin{center}
    {\Large\scshape Ensayo}\\[0.3em]
\end{center}

\vspace{0.1cm}

% Fecha
\begin{center}
    \small\scshape 06 de febrero de 2026
\end{center}

\vspace{0.2cm}

% Línea horizontal inferior (gruesa)
\noindent\rule{\textwidth}{1.2pt}

\vspace{0.6cm}

% ===== Caja de resumen =====

La física médica representa uno de los campos más interesantes de la física y uno de los puntos a destacar es que se centra en beneficiar a la salud. En el presente ensayo hablaré un poco sobre esta rama de la física y por qué es importante saber herramientas de programación en el desarrollo de la misma.

Como mencioné anteriormente, mi área de interés es la física médica, porque combina lo mejor de dos mundos: física y medicina moderna . con el fin de beneficiar la salud y encontrar tratamientos adecuados para cada paciente.

Algunas áreas de la física médica son la radioterapia, medicina nuclear, etc.En estas áreas no solo es importante saber física, sino también contar con habilidades computacionales avanzadas para el análisis de datos de los pacientes.


La programación en física médica es una herramienta esencial que aplica lenguajes como Python, C++ o MATLAB para procesar imágenes médicas, calcular dosimetría en radioterapia y realizar control de calidad de aceleradores lineales. Esta competencia permite automatizar la planificación de tratamientos, modelar interacciones de radiación y mejorar significativamente la precisión clínica.

En el campo de la \textbf{radioterapia}, la programación permite desarrollar algoritmos sofisticados para el cálculo de dosis y la optimización de planes de tratamiento. El objetivo es maximizar la dosis de radiación al tumor mientras se minimiza la exposición del tejido sano circundante, un balance crítico que puede marcar la diferencia en el éxito del tratamiento y la calidad de vida del paciente.


Finalmente, en \textbf{medicina nuclear}, la programación permite simular la distribución de radiofármacos en el cuerpo humano, ayudando a predecir y optimizar los tratamientos antes de su aplicación clínica.


El dominio de ciertos lenguajes de programación es esencial para un físico médico. \textbf{Python} se ha convertido en el lenguaje más utilizado en investigación y análisis de datos médicos gracias a sus poderosas librerías científicas como NumPy y SciPy, que facilitan el manejo de grandes volúmenes de datos y cálculos complejos.

\textbf{MATLAB} es ampliamente empleado en el procesamiento de señales e imágenes médicas, así como en la creación de interfaces gráficas para el análisis de datos, siendo especialmente valioso en entornos educativos y de investigación.

Por su parte, \textbf{C++} es frecuente en simulaciones de Monte Carlo para física de radiaciones, donde la eficiencia computacional es crítica debido a la complejidad de los cálculos involucrados en el modelado de interacciones de partículas con la materia.


La programación no es simplemente una herramienta complementaria en física médica, sino una competencia fundamental que define las capacidades y el alcance del profesional moderno en este campo. La formación en esta área, que típicamente se realiza a través de licenciaturas y maestrías en física médica o ingeniería biomédica, debe incluir necesariamente un componente robusto de programación.



\end{document}